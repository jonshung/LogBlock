\chapter{Introduction}

LogBlock is an online social media Service platform with a programming media sharing centrical view. The platform provides an easier and user-friendly means 
for both new and senior developers to share their open contributions without compromising too much on reachability of contents.

    \section{Purpose}
    The LogBlock Software Requirements Proposal is addressed to general team development members and business officers. 
    The specification provides a developement contract applied to all personel of the developement team.

    \section{Intended Use}
    The document serve as a production contract, and entails the developement procedures of the remaining project operations.
    It also serves as a guidance for \textbf{necessary actions and procedures} to execute during dispute involving contract violation.
    \\\\
    Albeit usually up-to-date, consumers of this document should note that due to the ever-changing requirements as 
    well as the production model of the development organization, minors changes are to be introduced to accomodate with the needed alternation associated in each 
    development cycle's retrospective. And thus, it is advise to \textbf{revise} the document in known interval.

    \section{Development Team Information}
    \begin{itemize}
        \item Hung K. Ngu (22127134) (nkhung22@clc.fitus.edu.vn)
        \item Long T. Tran (22127249) (ttlong221@clc.fitus.edu.vn)
        \item An C. Trinh (22127004) (tcan22@clc.fitus.edu.vn)
        \item Cuong Ng. N. Tran (22127048) (tnncuong22@clc.fitus.edu.vn)
        \item Long T. T. Nguyen (22127247) (nttlong22@clc.fitus.edu.vn) 
    \end{itemize}
\newpage

\chapter{Development Team Roles and Responsibilities}
Each team member should be responsible for the development of functionality descriptions as well as detailed
implementation of the whole system due to the small team size.
In order to provide a more reliable method of man­age­ment, the team role is defined as follow:
\begin{itemize}
    \item \textbf{Team Leader, Project Manager}: Hung K. Ngu
    \item \textbf{Executive Officer, Business Analyst}: Long T. Tran
    \item \textbf{Back-End Implementation Personel}: An C. Trinh
    \item \textbf{Front-End Implementation Personel}: Cuong Ng. N. Tran
    \item \textbf{Quality Assurance Personel}: Long T. T. Nguyen
\end{itemize}

\textit{Project Manager} should be responsible for the \textbf{team coordination, performance and development documentation}. 
\textit{Project Manager} is expected to apply development schemes and follows standard developement procedures in order to accurately
track and maintain production quality.
\textit{Project Manager} should be responsible for \textbf{reporting and finalizing} production results to User.
\\\\
\textit{Executive Officer} should be responsible for \textbf{team specific production and decision-making process}. Detailed development models is implemented 
based on Executive Office plan.\textit{ Executive Officer} is expected to \textbf{finalize each modular functionalities} of the product system specification.
\\\\
\textit{Business Analyst} provides the necessary technical \textbf{functional and non-functional requirements} for the development team, while ensuring the consistency of user needs
and constraints of the business domain.\textit{ Business Analyst} is expected to deliver \textbf{user requirements and system requirements} accordingly throughout the development process
of the project.
\\\\
\textit{Implementation Personel} is responsible for the specific \textbf{implementation of the codebase} as well as \textbf{unit testing} and ensuring working, performance and
competitive production results.\textit{ Implementation Personel} is expected to provide acceptable results in reasonable amount of time.
\\\\
\textit{Quality Assurance Personel} is expected to ensures the correctness and availability of the product operational functionalities requirements, as well as 
implementing testing suite while maintaining production results alignment with \textbf{Project Constraints}.

\chapter{Development Communication Plans}
The Development Team communication plans will be conducted \textbf{primarily online}, with the exception of conflict resolutions and contract changes to be decided in person.
Members of developement team should be responsible for their updates on the team's status and quota. An interval of two days is recommended in order to not miss crucial
informations and executive decisions.
\\\\
Team members are expected to be noticed of communication updates and provides reply if necessary in manners of at-most \textbf{three to five hours} since the message departure.

    \section{Communication Stack}
    Based on the scope of the project, the communication stack should not be too complicated and based around third party services, outlined as follow:
    \begin{itemize}
        \item \textbf{Primary Communication Channel}: Discord
        \item \textbf{Secondary Communication Channel}: Email
        \item \textbf{Project and Task Management Service}: Jira
    \end{itemize}

    \section{Executive Decision and Conflict Resolution Protocols}
    The Development Team will conduct \textbf{weekly retrospective survey} and forward-looking between \textbf{19th hour and 22nd hour} at weekend 
    in order to evaluate team's production result as well as planning the upcoming development direction; 
    finalizing production materials for submission at each 2-week sprints.
    \\\\
    In order to promote innovative ideas and quick iterative development, \textit{Scrum model} should be used as the engineering process. The team should be able
    to quickly go through with one's work without fear of failure. After a trial and error process, the finalized ideas that inlined with the \textbf{project core concept}
    are considered to be accepted into main development branch. 
    \\\\
    The process of deciding the integration of changes should be \textbf{held at all weekends}, during the retrospective and forward-looking process, in order to reflects
    changes induced by technical blocks or by other constraints introduced in the development sprint. The process is based on the constrast majority vote system, 
    enabling a quick and and thorough exhanges of ideas through continuous \textbf{feedback and demonstration}. After the demonstration section, \textit{final executive decisions are
    relied on Executice Officer for acceptence.}
    \\\\
    \\
    Conflicts are also part of the process, and undecisive actions of these aspects will cause long-term harm to the project development. Conflicts are to be \textbf{reported
    directly to Project Manager} as per this contract, and violations of contracts terms are also subjected to incident loggings. \textbf{The Conflicts Resolution Process} determines
    if it is necessary to escalate the actions needed.
    \\\\
    If the Conflicts are affecting \textit{less than three of the team's development member}, disputes will be handled internally
    by direct communication and case-specific actions. 
    \\\\
    Otherwise, if the Conflicts affect the \textit{majority of the team's developement member}, an in person meeting will be held
    to account for the related incidents and \textbf{consideration to contract changes} if it is a major blocking in the team's performance.

    \section{Work Schedules and Deadlines}
    The Development Team primarily operates online with \textbf{flexible schedule}, allowing other members to cover each's works while not compromising product quality
    and timing contraints.
    \\\\
    The Development Process is conducted every week, spanning from Monday to Sunday. \textit{The ending of each week} should be a milestone for product reanalysis and 
    performance assertion. Tasks of each Process iteration will be declared and is expected to be completed in a timely manners based on the given periods.
    \\\\
    \textbf{The Central Development Process Work Session} will be conducted from the \textbf{20th hour to 22nd hour on Sunday}, where the team will review production assets in order
    to apply Quality Controls and maintaining consistency between different codebases.
    \\\\
    In the case of missing deadlines, the entire development team is responsible to commit \textbf{mandatory overtime} of combined work in order to resolve missing deadlines.
    The responsible team member of the original task will be noticed of the incidents; an unreasonable amount of missing deadlines incidents will be reported to Project
    Manager for further accessment.

    \section{Code and Documentation Standards}
    Each individual development team member is responsible for following the Code of Conducting while practicing implementation. The specific coding convention
    is yet to be decided, but a good propotion will be based on \textbf{C++ Core Guidelines}.
    \\
    Individual development team member should prioritize \textbf{readability and maintainability} while implementing functionalities, obfuscation of codebase is yet to be
    necessary and is considered a bad practice in this project.
    \\\\
    Based on the scope of the project, the development operations stack should not be too complicated and based around third party Services, outlined as follow:
    \begin{itemize}
        \item \textbf{Development Operations and Source Control Service}: Github
        \item \textbf{Project Documents and Assets Storage}: Google Drive
    \end{itemize}

    Code testings software is yet to be determined. However, development team member should be responsible for quality working production results and self-penetration
    testing before deliveries.
    \\\\
    Before the finalized decision on core implementation technologies, documentation standards is expected to follow \textbf{IEEE Standard for Software User Documentation}.
    Based on chosen technology, a more flexible and open, \textit{autonomous solutions} can be used to automate the technical documentation process. Outside of implementation,
    the documentation is to follow the aforementioned standard.

    \section{Accountability and Performance}
    Development team members are expected to be on time with assigned tasks, while also maintaining quality in production results. In rare cases when a task period
    is misaligned, further set-back can be applied to accomodate the necessary tasks.
    \\\\
    Development team members performance will be measured based on their \textbf{communication participation, their accessment on project responsiblities and their ability to
    provide exceptional results}. 
    \\\\
    Though not necessary, development team member is also encouraged to propose changes to the development pipeline if see fits; but the 
    final decision still relies on both Executive Officer and Project Manager.
    \\\\
    Lack of development participation will results in \textbf{competency reassement}. The Development Team will try to establish common grounds and provides necessary sup­ports
    for each individuals during the handling of underperforming members.
    \\\\
    However, punishment of \textbf{course points partitioning degradation} will be considered in some cases.
    If incidents are repeated in an unreasonable amount, or conflicts causing blockage in the development process, \textbf{notification to course instructor}
    will be considered if applicable.
    \\\\
    Lack of team contract adhering will also inherently be processed based on the previously mentioned procedures. Though the Project Manager may consider
    alternative methods such as \textit{contract validation} in order to determine if the incident is the result of the incomplete nature of this contract or by the
    subjective reasoning of individual.